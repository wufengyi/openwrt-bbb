The Contiki build system is designed to make it easy to compile Contiki applications for different hardware platforms or into a simulation platform by simply supplying different parameters to the make command, without having to edit makefiles or modify the application code.

The file example project in examples/hello-\/world/ shows how the Contiki build system works. The hello-\/world.\+c application can be built into a complete Contiki system by running make in the examples/hello-\/world/ directory. Running make without parameters will build a Contiki system using the native target. The native target is a special Contiki platform that builds an entire Contiki system as a program that runs on the development system. After compiling the application for the native target it is possible to run the Contiki system with the application by running the file hello-\/world.\+native. To compile the application and a Contiki system for the E\+S\+B platform the command make T\+A\+R\+G\+E\+T=esb is used. This produces a hello-\/world.\+esb file that can be loaded into an E\+S\+B board.

To compile the hello-\/world application into a stand-\/alone executable that can be loaded into a running Contiki system, the command make hello-\/world.\+ce is used. To build an executable file for the E\+S\+B platform, make T\+A\+R\+G\+E\+T=esb hello-\/world.\+ce is run.

To avoid having to type T\+A\+R\+G\+E\+T= every time make is run, it is possible to run make T\+A\+R\+G\+E\+T=esb savetarget to save the selected target as the default target platform for subsequent invocations of make. A file called Makefile.\+target containing the currently saved target is saved in the project\textquotesingle{}s directory.

Beside T\+A\+R\+G\+E\+T= there\textquotesingle{}s D\+E\+F\+I\+N\+E\+S= which allows to set arbitrary variables for the C preprocessor in form of a comma-\/separated list. Again it is possible to avoid having to re-\/type i.\+e. D\+E\+F\+I\+N\+E\+S=M\+Y\+T\+R\+A\+C\+E,M\+Y\+V\+A\+L\+U\+E=4711 by running make T\+A\+R\+G\+E\+T=esb D\+E\+F\+I\+N\+E\+S=M\+Y\+T\+R\+A\+C\+E,M\+Y\+V\+A\+L\+U\+E=4711 savedefines. A file called Makefile.\+esb.\+defines is saved in the project\textquotesingle{}s directory containing the currently saved defines for the E\+S\+B platform.

Makefiles used in the Contiki build system The Contiki build system is composed of a number of Makefiles. These are\+:


\begin{DoxyItemize}
\item Makefile\+: the project\textquotesingle{}s makefile, located in the project directory.
\item Makefile.\+include\+: the system-\/wide Contiki makefile, located in the root of the Contiki source tree.
\item Makefile. (where  is the name of the platform that is currently being built)\+: rules for the specific platform, located in the platform\textquotesingle{}s subdirectory in the platform/ directory.
\item Makefile. (where  is the name of the C\+P\+U or microcontroller architecture used on the platform for which Contiki is built)\+: rules for the C\+P\+U architecture, located in the C\+P\+U architecture\textquotesingle{}s subdirectory in the cpu/ directory.
\item Makefile. (where  is the name of an application in the apps/ directory)\+: rules for applications in the apps/ directories. Each application has its own makefile.
\end{DoxyItemize}

The Makefile in the project\textquotesingle{}s directory is intentionally simple. It specifies where the Contiki source code resides in the system and includes the system-\/wide Makefile, Makefile.\+include. The project\textquotesingle{}s makefile can also define in the A\+P\+P\+S variable a list of applications from the apps/ directory that should be included in the Contiki system. The Makefile used in the hello-\/world example project looks like this\+: \begin{DoxyVerb}CONTIKI_PROJECT = hello-world
all: $(CONTIKI_PROJECT)

CONTIKI = ../..
include $(CONTIKI)/Makefile.include
\end{DoxyVerb}


First, the location of the Contiki source code tree is given by defining the C\+O\+N\+T\+I\+K\+I variable. Next, the name of the application is defined. Finally, the system-\/wide Makefile.\+include is included.

The Makefile.\+include contains definitions of the C files of the core Contiki system. Makefile.\+include always reside in the root of the Contiki source tree. When make is run, Makefile.\+include includes the Makefile. as well as all makefiles for the applications in the A\+P\+P\+S list (which is specified by the project\textquotesingle{}s Makefile).

Makefile., which is located in the platform// directory, contains the list of C files that the platform adds to the Contiki system. This list is defined by the C\+O\+N\+T\+I\+K\+I\+\_\+\+T\+A\+R\+G\+E\+T\+\_\+\+S\+O\+U\+R\+C\+E\+F\+I\+L\+E\+S variable. The Makefile. also includes the Makefile. from the cpu// directory.

The Makefile. typically contains definitions for the C compiler used for the particular C\+P\+U. If multiple C compilers are used, the Makefile. can either contain a conditional expression that allows different C compilers to be defined, or it can be completely overridden by the platform specific makefile Makefile.. 